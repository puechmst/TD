
\documentclass[a4paper]{article}
\usepackage[utf8]{inputenc}
\usepackage[T1]{fontenc}

\title{Programmation CUDA : \\
la convolution}
\date{}

\begin{document}
\maketitle
\section{Prise en main de CUDA}
En s'inspirant de ce qui a été présenté en cours, écrire un noyau de calcul renvoyant, à partir d'un vecteur
passé en paramètre, le vecteur formé en élevant chaque élément au carré. En déduire un noyau permettant de calculer
la norme euclidienne d'un vecteur passé en paramètre.
\section{convolution}
La convolution est une opération de base en traitement du signal et en intelligence artificielle connexionniste. 
Les premières couches d'un réseau de neurones profond réalisent souvent cette opération afin d'effectuer un filtrage 
des données. Si l'on se donne un vecteur d'entrée $X=(x_0, \dots, x_{n-1})$ et un vecteur de coefficients 
$H = (h_0, \dots, h_{p-1}), p \leq n$, le vecteur convolué  $Y = X \star H$ se définit comme suit~:
\begin{equation}
    \label{eq:convolution}
    y_j = \sum_{k=\max(0,j-p)}^j x_k h_{j-k}, \, j=0 \dots n-1
\end{equation}
Dans le cas bidimensionnel, $X$ et $H$ sont des matrices de dimensions respectives $m\times n$ et $p \times q$ et 
l'équation de convolution $Y=X \star H$ devient~:
\begin{equation}
    \label{eq:conv2D}
    y_{i,j} = \sum_{k=\max(0,i-p)}^i \sum_{l=\max(0,j-q)}^j x_{i,j} h_{i-k,j-l}, \, i=O\dots m-1, \, j=0\dots n-1
\end{equation}
\subsection{Approche naïve}
\subsubsection{Noyau de calcul}
En s'inspirant de ce qui a été vu en cours pour le produit matriciel, proposer un noyau CUDA permettant de calculer
le produit de convolution de deux vecteurs et de deux matrices. Les signatures des deux fonctions seront~: 
\begin{tabular}{c}
\leftline{\scriptsize\texttt{\_\_global\_\_void conv1D(int n, float *x, int p, float *h, float *y);}} \\
\leftline{\scriptsize\texttt{\_\_global\_\_ void conv2D(int m, int n, float *x, int p, int q, float *h, float *y);}}
\end{tabular}
\subsubsection{Appel du noyau}
Tester le bon fonctionnement du noyau en générant des vecteurs et des matrices aléatoires et en comparant avec les 
résultats obtenus sur CPU. Déterminer approximativement les performances en GFLOPS du produit de convolution sur CPU et GPU.
\subsection{Codage par blocs}
On peut améliorer les performances avec la mémoire partagée. Proposez un nouveau noyau exploitant au mieux la 
localité des données et comparez avec l'approche naïve.
\end{document}